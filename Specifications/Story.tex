\documentclass[a4paper,12pt]{book}

\usepackage[latin1]{inputenc}
\usepackage[T1]{fontenc}
\usepackage[english]{babel}
\usepackage{url}
\usepackage{array}
\usepackage{lmodern}
\usepackage{fancyhdr}
\usepackage{graphicx}

\pagestyle{plain}

\title{\Huge{Klifehrian} \\ \large{Story}}
\date{\textbf{Last update :} \today}
%-------------------------------------------------------------------------------------------
%-------------------------------------------------------------------------------------------
%-------------------------------------------------------------------------------------------
\begin{document}
	\maketitle
	\thispagestyle{empty}
	\setcounter{page}{0}
	%-------------------------------------------------------------------------------------------
	\section{Disclaimer}
		\paragraph{} This document is filled to the brim (and then some) with spoilers. You have been warned.
		
		\paragraph{} ``Klifehrian'' is both the name of the game and the name of the continent on which the story takes place. For the sake of simplicity, the former will always be referred to as ``The Game'', while the latter will stay as is.
	%-------------------------------------------------------------------------------------------
	\part{Backstory}
		\chapter{Creation} 
			The world in which Klifehrian resides was created by a group of 5 gods: Rufigh, Zelos, Fuhius, Ligam and Turmof. Each of them had participated in one way or another to the creation of this world, and they all had their own goals with it.
			\begin{itemize}
				\item Rufigh was a gardener at heart; he molded the lands and shaped the humans.
				\item Fughius worked on the inner side of the world. Her forge, located in the core of the world, is where she created most of the creatures inhabiting the world.
				\item Zelos was a free spirit, and wanted this world to be that way as well. The winds blowing accross the lands and oceans are a representation of this. It is said that it is him who breathed life into the humans.
				\item Ligam liked to call herself the Goddess of Fortune. Random acts of miracles were her trademark.
				\item Turmof was her counterpart; It is believed he is the one responsible for vices. Although he is not inherantly bad, his interest for the human race and its resilience can get slightly overboard.
			\end{itemize}
		\chapter{Destruction}
			\paragraph{} After creating this world, the gods had forgotten an important point: if you don't make your presence clear, you will be forgotten. That is what happened to them. Furious, Rufigh, Fughius and Zelos wanted to wipe them all out so they could restart. However, a forgotten god is a weakened god, and they couldn't manage to go all the way. Furthermore, they used their last ressources in trying to do so. Fughius went to sleep eternally in her forge. It is said Rufigh became a giant tree, though no one has ever seen one that could be him. It's Zelos who paid the biggest toll, as his body was scattered accross the winds he had created himself.
			\paragraph{} Ligam and Turmof, however, did not participate in this destruction attempt, and as such, survived. They stayed to look over the world; both of them shared a great interest for those humans.
			\paragraph{} Also to be noted: the giant spell the gods tried to cast wasn't without aftereffects. Leftovers traces of its destructive nature started to affect some of the living creatures, mutating them in beings driven by desctruction. Seeing there an opportunity to test humanity, Turmof gathered some of these traces and used them to create a demon from a human champion. After naming him Zackaria, he left him to his own devices, curious about the results.
	\part{Story}
		\chapter{Party time}
\end{document}